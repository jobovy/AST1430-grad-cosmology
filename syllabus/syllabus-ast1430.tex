\documentclass{article}

\usepackage{float}
\usepackage{hyperref}
\usepackage{url}
\usepackage{amsmath,amssymb}

\include{vc}

\pagestyle{empty}

\baselineskip 18pt
\textwidth 6.25in
\textheight 8.5in
\oddsidemargin 0.1in
\evensidemargin 0.1in
\marginparwidth 0in
\marginparsep 0in
\topmargin -.5in
\topskip -.5in
\parindent 0in
\parskip 1pt

\begin{document}

\begin{center}
  \LARGE{\scshape{AST1430: Cosmology}}\\[5pt]
  \Large{\scshape{Winter 2023}}\\[5pt]
  \large{(last updated: \today; rev. \githash)}\\[25pt]
\end{center}

\section*{Course description}

The goal of this course is to provide a more complete coverage of cosmology, 
and to develop concepts to the point of calculation. The topics to be covered 
include a brief introduction to relevant concepts from General Relativity, the 
model of an isotropic, homogeneous, expanding Universe, inflation, the origin 
and nature of the Cosmic Microwave background, Big-Bang Nucleosynthesis and 
baryogenesis, dark matter, linear perturbation theory, large-scale structure 
beyond linear perturbation theory, dark energy, and a discussion of the main 
observational cosmological probes.


\section*{Logistics}

See the README file of this repository.

\section*{Learning objectives}

Students will develop an understanding of diverse concepts in cosmology, and 
the physics that governs them. Specifically, the objectives are

\begin{itemize}

\item Understanding and being able to explain the physical principles that are 
  relevant for the large scale structure and evolution of the Universe from the 
  Big Bang until the present;

\item Being able to derive equations for the evolution of all of the major 
  components of the Universe (baryons, dark matter, dark energy, photons) from 
  the basic physical principles
\item Solving these equations analytically and numerically, sometimes with the 
  aid of basic mathematical software like Mathematica or Python packages such 
  as numpy or scipy;
\item Understanding how the cosmological model is observationally constrained 
  and how we that the Universe is dominated by dark energy and dark matter

\end{itemize}

\section*{Reading}

The main material will be presented in the slides and, in the second half, in 
a set of lecture notes. Additionally, the following books are recommended for
further reading:

\begin{itemize}

  \item Malcolm Longair, \emph{Galaxy Formation},
    2007, Springer.

  \item John Peacock, \emph{Cosmological Physics},
    1998, Cambridge University Press.

  \item Mo, van den Bosch, \& White, \emph{Galaxy Formation and Evolution},
    2010, Cambridge University Press. Errata can be found
    \href{http://people.umass.edu/hjmo/book/errata.pdf}{here}.

  \item Dodelson \& Schmidt, \emph{Modern Cosmology},
    2020, Academic Press.

\end{itemize}

\section*{Grading scheme}

\begin{itemize}

  \item {\bf Assignments:} 40\,\% over three assignments; see course website for due dates.

  \item {\bf Participation:} 20\,\%

  \item {\bf Presentations:} 20\,\%

  \item {\bf Take-home final + oral exam:} 20\,\%

You are allowed to (and are encouraged to!) work together with
classmates on the assignments, but each student must hand in an
independent write-up of their solutions. The take-home final should be
your own work. Solutions must be written up in a detailed enough
manner to demonstrate that you understand each step. The oral exam
will consist of a discussion of the take-home final with follow-up
questions.

\end{itemize}

\section*{Academic integrity}

From Appendix D of the Academic Integrity Handbook:
\begin{quote}
  Academic integrity is one of the cornerstones of the University of
  Toronto. It is critically important both to maintain our community
  which honours the values of honesty, trust, respect, fairness, and
  responsibility and to protect you, the students within this
  community, and the value of the degree towards which you are all
  working so diligently.  

  According to Section B of the University of
  Toronto's Code of Behaviour on Academic Matter
  (\url{http://www.governingcouncil.utoronto.ca/policies/behaveac.htm})
  which all students are expected to read and by which they are
  expected to abide, it is an offence for students to:
  \begin{itemize}
    \item Use someone else's ideas or words in their own work without
      acknowledging explicitly that those ideas/words are not their
      own with a citation and quotation marks, i.e. to commit
      plagiarism.
  \item Include false, misleading, or concocted citations in their
    work.
  \item Obtain unauthorized assistance on any assignment. 
  \item Provide unauthorized assistance to another students. This
    includes showing another student your own work.
  \item Submit their own work for credit in more than one course
      without the permission of the instructors.
  \end{itemize}

  There are other offenses covered under the Code, but these are the
  most common. You are instructed to respect these rules and the
  values that they protect.
\end{quote}

\section*{Schedule}

\begin{itemize}

  \item {\bf Week 1:} Class logistics, introduction, basic observations.

  \item {\bf Week 2:} Basic GR, RW metric, Distances, coordinates, Friedmann 
    equations.

  \item {\bf Week 3:} Cosmological models, consistency with observations, 
  early hot Universe, BBN.

  \item {\bf Week 4:} Inflation, perturbations \& structure pre-recombination.

  \item {\bf Week 5:} CMB: basics, polarization, secondaries.

  \item {\bf Week 6:} Early-universe presentations.

  \item {\bf Week 7:} Post-recombination growth of structure, formation of dark 
  matter halos, halo mass function.

  \item {\bf Week 8:} The relation between dark matter halos and galaxies.

  \item {\bf Week 9:} Probing the cosmic density field / clustering.

  \item {\bf Week 10:} Late-universe presentations; late-time cosmological 
  observations: BAO, supernovae, weak lensing, etc.

  \item {\bf Week 11:} The H0 controversy: how fast exactly is the Universe 
  expanding today?

  \item {\bf Week 12:} Review.

\end{itemize}

\end{document}

